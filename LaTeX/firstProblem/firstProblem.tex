\documentclass[a4paper,11pt]{article}
\usepackage{amsmath}
\usepackage{amssymb}
\oddsidemargin=-5.00mm
\textwidth=170.00mm
\topmargin=-10.00mm
\textheight=240.00mm

\begin{document}
% define the title
\author{Tom Dolan}
\title{First Problem in \LaTeX}
\date{18/12/2013}
\maketitle
\newpage

\section{First Problem}
\setlength{\unitlength}{4mm}
\begin{picture}(35,12)
\put(2.5,2){\line(1,0){35}}
\put(2.5,12){\line(1,0){35}}
\put(2.5,2.2){\line(1,0){35}}
\put(2.5,11.8){\line(1,0){35}}
\put(5,5){\vector(1,0){3}}
\put(5,5){\vector(0,1){3}}
\put(4,7){$y$}
\put(7,4){$x$}
\put(38,2){$y = 0$}
\put(38,12){$y = L$}
\end{picture}
\begin{align*}
\nabla\cdot\left(\frac{1}{\rho}\nabla P\right) + \frac{\omega^2}{B}P & = 0 \\
\text{So} \qquad \nabla^2 P + k^2 P & = 0 \\
\frac{\partial^2 P}{\partial x^2} + \frac{\partial^2 P}{\partial y^2} + k^2 P & = 0\\\\
\text {So if $P = X(x)Y(y)$, then }\nabla^2 P & = X''Y + XY''\\
\text{So} \qquad \frac{X''}{X} + \frac{Y''}{Y} + k^2 & = 0\\
i.e. \qquad \frac{X''}{X} + k^2 = -\frac{Y''}{Y} & = \alpha^2\\\\
\text{So we have two 2nd order linear ODEs:}\\
\frac{X''}{X} + k^2 &= \alpha^2\\
X'' + (k^2 - \alpha^2) X & = 0\tag{1}\\ \\
\text{and}\qquad\qquad -\frac{Y''}{Y} & = \alpha^2\\
Y'' + \alpha^2Y & = 0\tag{2}\\\\
\text{Solving (2) gives:}\qquad\\
Y = a_1cos(\alpha y) + a_2sin(\alpha y)\\\\
\text{And given Neumann boundary conditions:}\\
\frac{\partial P}{\partial \overrightarrow{n}} = 0 \quad \forall \bold{x}\in \partial P
\implies & (\frac{\partial P}{\partial x}, \frac{\partial P}{\partial y})\cdot(0,1) = 0\\
\implies & \frac{\partial P}{\partial y} = 0 \quad \text{for } y = 0,L\\
\implies & Y'(0) = Y'(L) = 0\\
\\
Y'(y) = -\alpha a_1\sin(\alpha y) + \alpha a_2\cos(\alpha y)\\
\text{So} \qquad  Y'(0) = \alpha a_2 = 0  \implies & a_2 = 0\\
\text{and} \qquad  Y'(L) =  -\alpha a_1\sin(\alpha y) = 0 \implies & \alpha = n\pi/L \qquad \text{for } n \in \mathbb{N}\\\\
\end{align*}

$\text{So as there are still unknown constants in X, set } Y_n = \cos(\frac{n\pi y}{L})$\\

Now with $\alpha_n  = n\pi/L$, (1) becomes:
\begin{align*}
X'' + (k^2 - \left(\frac{n\pi}{L}\right)^2)X & = 0 \qquad \forall n \in \mathbb{N}\\
\text{so} \qquad X_n & = b_{n+}e^{i\beta_n x} + b_{n-}e^{-i\beta_n x} \qquad \left(\text{where $\beta_n = \sqrt{k^2 - \left(\frac{n\pi}{L}\right)^2}$}\right)\\
\text{so}\qquad  P_n = X_nY_n & = \left(b_{n+}e^{i\beta_n x} + b_{n-}e^{-i\beta_n x}\right)\cos(\alpha_n y)\\
\text{set } k = 1 \implies P_n & = \left(b_{n+}e^{i\sqrt{1 - \left(\frac{n\pi}{L}\right)^2}x} + b_{n-}e^{-i\sqrt{1 - \left(\frac{n\pi}{L}\right)^2}x}\right)\cos(\frac{n\pi y}{L})\\\\
\text{Now if we write} \qquad Y_n & = \cos(\alpha_ny) = \frac{1}{2}\left(e^{i\alpha_ny}+e^{-i\alpha_ny}\right)\\
\text{So } P_n & = X_nY_ne^{-i\omega t}\\
& = \frac{1}{2}\left(b_{n+}e^{i\beta_nx}+b_{n-}e^{-i\beta_nx}\right)\left(e^{i\alpha_ny}+e^{-i\alpha_ny}\right)e^{-i\omega t} \\
& = \frac{1}{2}\Big(b_{n+}e^{i\beta_nx + i\alpha_ny - i\omega t} + b_{n+}e^{i\beta_nx - i\alpha_ny - i\omega t} \\
& \quad+ b_{n-}e^{-i\beta_nx + i\alpha_ny - i\omega t} + b_{n-}e^{-i\beta_nx - i\alpha_ny - i\omega t}\Big) \\
\end{align*}
Now, $\alpha_n = \frac{n\pi}{L}$ and $\beta_n = \sqrt{1-\left(\frac{n\pi}{L}\right)^2} = \sqrt{1-\alpha_n^2}$.  What we want is for terms of $P_n$ to be of the form $ae^{i\cos(\psi)x + i\sin(\psi)y - i\omega t}$ as this is a simple plane wave traveling at an angle of $\psi$ to the x axis
\begin{align*}
\text{so if }\quad \alpha_n & = \sin(\psi_n) \qquad \implies \alpha_n<1\\
\text{then } \quad \beta_n & = \sqrt{1-\alpha_n^2}\\
& = \sqrt{1-sin(\psi_n)^2}\\
& = \cos(\psi_n) \qquad \text{for } -\pi/2 \ge \psi_n \ge \pi/2\\\\
\implies P_n & =  \frac{1}{2}\Big(b_{n+}e^{i\cos(\psi_n)x + i\sin(\psi_n)y - i\omega t} + b_{n+}e^{i\cos(\psi_n)x - i\sin(\psi_n)y - i\omega t} \\
& \quad+ b_{n-}e^{-i\cos(\psi_n)x + i\sin(\psi_n)y - i\omega t} + b_{n-}e^{-i\cos(\psi_n)x - i\sin(\psi_n)y - i\omega t}\Big) \\
& = \frac{1}{2}\Big(b_{n+}e^{i\cos(\psi_n)x + i\sin(\psi_n)y - i\omega t} + b_{n+}e^{i\cos(-\psi_n)x + i\sin(-\psi_n)y - i\omega t} \\
& \quad+ b_{n-}e^{-(i\cos(-\psi_n)x + i\sin(-\psi_n)y + i\omega t)} + b_{n-}e^{-(i\cos(\psi_n)x + i\sin(\psi_n)y + i\omega t)}\Big) \\
\end{align*}
So if $b_{n+} = 1$ and $b_{n-} = 0$ and if $n<L/\pi$ then:\\
$P_n = \frac{1}{2}\Big(e^{i\cos(\psi_n)x + i\sin(\psi_n)y - i\omega t} + e^{i\cos(-\psi_n)x + i\sin(-\psi_n)y - i\omega t}\Big)$ which is the sum of two plane waves traveling at angles of $\psi_n$ and $-\psi_n$ to the x axis, with $-\pi/2\ge\psi_n\ge\pi/2$. This corresponds to a traveling wave with nodal lines, where $P_n = 0$ and $t = 0$ at:
\begin{align*}
x & = \frac{\pi(2l-1)}{2\cos(\psi_n)} \qquad \text{for } l \in \mathbb{Z}\\
\text{and} \qquad y & = \frac{\pi(2k-1)}{2\sin(\psi_n)} \qquad \text{for } k = 1, \ldots, n \text{ and } n>0\\
\end{align*}
For $n = 0$ there are no nodal lines in the $x$ direction as the two plane waves are equal so the resultant wave is the same plane wave.\\\\
As $|e^{i\cos(\psi_n)x + i\sin(\psi_n)y - i\omega t}|$ and $|e^{i\cos(-\psi_n)x + i\sin(-\psi_n)y - i\omega t}|$ are bounded by 1, half their sum, i.e. $|P|$ is bounded by 1. \\\\
For $\alpha_n>1 $ we have $ \beta_n = \sqrt{1-\alpha_n^2} = i\sqrt{\alpha_n^2 - 1} = i\gamma_n$ where $\gamma_n$ is some real constant. 
So for $b_{n+} = 1$ and $b_{n-} = 0$:
\[P_n =  \frac{1}{2}\Big(e^{-\gamma_nx + i\alpha_ny - i\omega t} +e^{-\gamma_nx - i\alpha_ny - i\omega t}\Big)\]
Which has no wave property in the x direction but now has exponential growth/decay. 

\section{Boundary Conditions}
\[P_{n\pm}(x,y) = b_{n\pm}e^{\pm i\beta_nx}\cos(\frac{n\pi y}{L})\]
\[P(x,y) = \sum_{n\in\mathbb{N}}P_{n+}(x,y) + P_{n-}(x,y)\]
If we add two more boundary conditions we can solve for the final two constants, so set $P(a,y) = f(y)$ and $P(b,y) = g(y)$. 
so at $x = a$:
\begin{align*}
P(a,x)  = & \sum_{n\in\mathbb{N}}P_{n+}(a,y) + P_{n-}(a,y) = f(y)\\
\implies & \sum_{n\in\mathbb{N}}\left(b_{n+}e^{i\beta_na} + b_{n-}e^{-i\beta_na}\right)\cos(\frac{n\pi y}{L}) = f(x)\\
\text{So } & \sum_{n\in\mathbb{N}}A_n\cos(\frac{n\pi y}{L}) = f(x)\\
\end{align*}
which is the fourier series for $f(x)$
\[\text{So}\qquad A_0 = \frac{1}{L}\int^L_0f(y)dy \qquad \text{and} \qquad A_{m \ne 0} = \frac{2}{L}\int^L_0f(y)\cos(\frac{m\pi y}{L})dy\]
And similarly for $B_m$ at $x = b$\\\\
Now: 
\[\qquad b_{n+}e^{i\beta_na} + b_{n-}e^{-i\beta_na} = A_n \qquad \text{and}\qquad b_{n+}e^{i\beta_nb} + b_{n-}e^{-i\beta_nb} = B_n\]
which is just a system of two equations with two unknowns

\end{document}}

