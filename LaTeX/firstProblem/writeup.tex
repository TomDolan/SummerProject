\documentclass[a4paper,11pt]{article}
\usepackage{amsmath}
\usepackage{amssymb}
\usepackage{graphicx}
\oddsidemargin=-5.00mm
\textwidth=170.00mm
\topmargin=-10.00mm
\textheight=240.00mm

\begin{document}
% define the title
\author{Tom Dolan}
\title{Wave Propagation Through a Waveguide}
\date{18/12/2013}
\maketitle
\newpage

\section{Introduction}
Initially I am looking at a wave traveling through a simple waveguide comprised of two parallel boundaries at $y = 0$ and $y = L$ with Neumann boundary conditions. \\

\setlength{\unitlength}{4mm}
\begin{picture}(35,12)
\put(2.5,2){\line(1,0){35}}
\put(2.5,12){\line(1,0){35}}
\put(2.5,2.2){\line(1,0){35}}
\put(2.5,11.8){\line(1,0){35}}
\put(5,5){\vector(1,0){3}}
\put(5,5){\vector(0,1){3}}
\put(4,7){$y$}
\put(7,4){$x$}
\put(38,2){$y = 0$}
\put(38,12){$y = L$}
\end{picture}\\
This can be represented by solving
\[\nabla\cdot\left(\frac{1}{\rho}\nabla P(x,y)\right) + \frac{\omega^2}{B}P(x,y) = 0 \]
Where $P(x,y)$ is the pressure inside the waveguide at the point $(x,y)$. As the mass density $\rho$, the angular frequency $\omega$ and the bulk modulus $B$ are all constants the equation becomes:
\[ \nabla^2 P + k^2 P = 0 \qquad \text{where } k^2 = \frac{\rho\omega^2}{B}\]
Now seeking a solution of the form $P= X(x)Y(y)$  we get two 2nd order linear ODEs:
\begin{align*}
X'' + (k^2 - \alpha^2) X & = 0\tag{1}\\
\text{and } \qquad Y'' + \alpha^2Y & = 0\tag{2}\\
\end{align*}
\begin{align*}
\text{Solving (2) gives:}\qquad\\
Y = a_1\cos(\alpha y) + a_2\sin(\alpha y)\\\\
\text{And given Neumann boundary conditions:}\\
\frac{\partial P}{\partial \overrightarrow{n}} = 0 \quad \forall \bold{x}\in \partial P
\implies & (\frac{\partial P}{\partial x}, \frac{\partial P}{\partial y})\cdot(0,1) = 0\\
\implies & \frac{\partial P}{\partial y} = 0 \quad \text{for } y = 0,L\\
\implies & Y'(0) = Y'(L) = 0\\
\\
Y'(y) = -\alpha a_1\sin(\alpha y) + \alpha a_2\cos(\alpha y)\\
\text{So} \qquad  Y'(0) = \alpha a_2 = 0  \implies & a_2 = 0\\
\text{and} \qquad  Y'(L) =  -\alpha a_1\sin(\alpha y) = 0 \implies & \alpha = n\pi/L \qquad \text{for } n \in \mathbb{N}\\
\end{align*}
So as there are still unknown constants in $X$, set $Y_n = \cos(\frac{n\pi y}{L})$

Now with $\alpha_n  = n\pi/L$, (1) becomes:
\begin{align*}
X_n'' + (k^2 - \left(\frac{n\pi}{L}\right)^2)X_n & = 0 \qquad \forall n \in \mathbb{N}\\
\text{so} \qquad X_n & = b_{n+}e^{i\beta_n x} + b_{n-}e^{-i\beta_n x} \qquad \left(\text{where $\beta_n = \sqrt{k^2 - \left(\frac{n\pi}{L}\right)^2}$}\right)\\
\text{so}\qquad  P_n = X_nY_n & = \left(b_{n+}e^{i\beta_n x} + b_{n-}e^{-i\beta_n x}\right)\cos(\alpha_n y)\\
 \end{align*}
And as the PDE is linear the full solution is the sum of all $P_n$
\[i.e.\qquad P = \sum_{n\in\mathbb{N}}P_n = \sum_{n\in\mathbb{N}}X_nY_n\]

\section{properties of $P_n$}
\begin{align*}
\text{If we write} \qquad Y_n & = \cos(\alpha_ny) = \frac{1}{2}\left(e^{i\alpha_ny}+e^{-i\alpha_ny}\right)\\
P_n & = X_nY_ne^{-i\omega t}\\
& = \frac{1}{2}\Big(b_{n+}e^{i\beta_nx + i\alpha_ny - i\omega t} + b_{n+}e^{i\beta_nx - i\alpha_ny - i\omega t} \\
& \quad + b_{n-}e^{-(i\beta_nx + i\alpha_ny + i\omega t)}+ b_{n-}e^{-(i\beta_nx - i\alpha_ny + i\omega t)}\Big) \\
\text{Now let } \qquad P_{n+} & = \frac{b_{n+}}{2}\Big(e^{i\beta_nx + i\alpha_ny - i\omega t} + e^{i\beta_nx - i\alpha_ny - i\omega t}\Big) \\
\text{And} \qquad P_{n-} & = \frac{ b_{n-}}{2}\Big(e^{-(i\beta_nx + i\alpha_ny + i\omega t)}+e^{-(i\beta_nx - i\alpha_ny + i\omega t)}\Big) \\
\end{align*}
$P_{n+}$ is a wave traveling left along the x axis and $P_{n-}$ is the same wave traveling right. The two waves act exactly the same so we will look only at $P_{n+}$\\\\
What we want is for terms of $P_{n+}$ to be of the form $e^{i\cos(\psi)x \pm i\sin(\psi)y - i\omega t}$ as this is a simple plane wave traveling at an angle of $\psi$ to the x axis. Now, for k = 1, $\alpha_n = \frac{n\pi}{L}$ and $\beta_n = \sqrt{1-\left(\frac{n\pi}{L}\right)^2} = \sqrt{1-\alpha_n^2}$.  
\begin{align*}
\text{so if }\quad \alpha_n & = \sin(\psi_n) \qquad \implies \alpha_n\le1\\
\text{then } \quad \beta_n & = \sqrt{1-\alpha_n^2}\\
& = \sqrt{1-sin(\psi_n)^2}\\
& = \cos(\psi_n) \qquad \text{for } -\pi/2 \ge \psi_n \ge \pi/2\\\\
\implies P_n & =  \frac{1}{2}\Big(b_{n+}e^{i\cos(\psi_n)x + i\sin(\psi_n)y - i\omega t} + b_{n+}e^{i\cos(\psi_n)x - i\sin(\psi_n)y - i\omega t}\Big) \\
& = \frac{1}{2}\Big(b_{n+}e^{i\cos(\psi_n)x + i\sin(\psi_n)y - i\omega t} + b_{n+}e^{i\cos(-\psi_n)x + i\sin(-\psi_n)y - i\omega t}\Big) \\
\end{align*}
So $P_{n+} = \frac{b_{n+}}{2}\Big(e^{i\cos(\psi_n)x + i\sin(\psi_n)y - i\omega t} + e^{i\cos(-\psi_n)x + i\sin(-\psi_n)y - i\omega t}\Big)$ which is the sum of two plane waves traveling at angles of $\psi_n$ and $-\psi_n$ to the x axis, with $-\pi/2\ge\psi_n\ge\pi/2$ (figure~\ref{fig:twowaves}). 
\begin{figure}[h]
    \centering
    \includegraphics[width=0.4\textwidth]{4}
    \includegraphics[width=0.4\textwidth]{5}
    \caption{two plane waves traveling at angles of $\psi$ and $-\psi$ to the $x$ axis.}
    \label{fig:twowaves}
\end{figure}
This corresponds to (figure~\ref{fig:combination}), a traveling wave with nodal lines, along which $P_n = 0$ initially at:
\begin{align*}
x & = \frac{\pi(2l-1)}{2\cos(\psi_n)} \qquad \text{for } l \in \mathbb{Z}\\
\text{and} \qquad y & = \frac{\pi(2k-1)}{2\sin(\psi_n)} \qquad \text{for } k = 1, \ldots, n \text{ and } n>0
\end{align*}
\begin{figure}[h]
    \centering
    \includegraphics[width=0.8\textwidth]{1}
    \caption{Combination of the two plane waves}
    \label{fig:combination}
\end{figure}

For $n = 0$ there are no nodal lines in the $x$ direction as the two plane waves are equal so the resultant wave is the same plane wave traveling left along the x axis (figure~\ref{fig:psi0}).\\\\
\begin{figure}[h]
    \centering
    \includegraphics[width=0.8\textwidth]{6}
    \caption{$\psi = 0$, corresponds to a wave traveling left along the $x$ axis}
    \label{fig:psi0}
\end{figure}

As $|e^{i\cos(\psi_n)x + i\sin(\psi_n)y - i\omega t}|$ and $|e^{i\cos(-\psi_n)x + i\sin(-\psi_n)y - i\omega t}|$ are bounded by 1, $|P_{n+}|$ is bounded by $b_{n+}$. \\\\
For $\alpha_n\ge1 $ we have $ \beta_n = \sqrt{1-\alpha_n^2} = i\sqrt{\alpha_n^2 - 1} = i\gamma_n$ where $\gamma_n$ is a real constant. so:
\[P_{n+} =  \frac{b_{n+}}{2}\Big(e^{-\gamma_nx + i\alpha_ny - i\omega t} +e^{-\gamma_nx - i\alpha_ny - i\omega t}\Big) = \frac{b_{n+}}{2}e^{-\gamma_nx}\Big(e^{i\alpha_ny - i\omega t} +e^{- i\alpha_ny - i\omega t}\Big)\]
Which has no wave property in the x direction but now has exponential growth/decay (figure~\ref{fig:exponential}). 
\begin{figure}[h]
    \centering
    \includegraphics[width=0.8\textwidth]{7}
    \caption{$\alpha \ge 1$, the $X$ component is now exponential}
    \label{fig:exponential}
\end{figure}

\section{Boundary Conditions}
\[P_{n\pm}(x,y) = b_{n\pm}e^{\pm i\beta_nx}\cos(\frac{n\pi y}{L})\]
\[P(x,y) = \sum_{n\in\mathbb{N}}P_{n+}(x,y) + P_{n-}(x,y)\]
If we add two more boundary conditions we can solve for the final two constants, so set $P(a,y) = f(y)$ and $P(b,y) = g(y)$. 
so at $x = a$:
\begin{align*}
P(a,x)  = & \sum_{n\in\mathbb{N}}P_{n+}(a,y) + P_{n-}(a,y) = f(y)\\
\implies & \sum_{n\in\mathbb{N}}\left(b_{n+}e^{i\beta_na} + b_{n-}e^{-i\beta_na}\right)\cos(\frac{n\pi y}{L}) = f(x)\\
\text{So } & \sum_{n\in\mathbb{N}}A_n\cos(\frac{n\pi y}{L}) = f(x)\\
\end{align*}
which is the fourier series for $f(x)$
\[\text{So}\qquad A_0 = \frac{1}{L}\int^L_0f(y)dy \qquad \text{and} \qquad A_{m \ne 0} = \frac{2}{L}\int^L_0f(y)\cos(\frac{m\pi y}{L})dy\]
And similarly for $B_m$ at $x = b$\\\\
Now: 
\[\qquad b_{n+}e^{i\beta_na} + b_{n-}e^{-i\beta_na} = A_n \qquad \text{and}\qquad b_{n+}e^{i\beta_nb} + b_{n-}e^{-i\beta_nb} = B_n\]
which is just a system of two equations with two unknowns which is easily approximated numerically. \\

The simplest example, setting the boundary conditions to be constant, means only the first mode's amplitude will be nonzero in the solution (figure~\ref{fig:f0g1}).
\begin{figure}[h]
    \centering
    \includegraphics[width=0.8\textwidth]{11}
    \caption{Boundary conditions $f(y) = 0$, $g(y) = 1$}
    \label{fig:f0g1}
\end{figure}
However, more complicated and not necessarily continuous functions are also valid boundary conditions (figure~\ref{fig:fsingstep}). 
\begin{figure}[h]
    \centering
    \includegraphics[width=0.45\textwidth]{8}
    \includegraphics[width=0.45\textwidth]{9}
    \caption{In this case $f(y)$ is a sin wave, $g(y)$ is a step function}
    \label{fig:fsingstep}
\end{figure}

\begin{thebibliography}{9}

\bibitem{lamport94}
    Agn\`es Maurel, Jean-Fran\c cois Mercier, Simon F\'elix
    \emph{Wave propagation through penetrable scatterers in a waveguide and through a penetrable grating}.
    2013

\end{thebibliography}
\end{document}}

