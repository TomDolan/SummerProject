\documentclass[a4paper,11pt]{article}
\usepackage{amsmath}
\usepackage{amssymb}
\oddsidemargin=-5.00mm
\textwidth=170.00mm
\topmargin=-10.00mm
\textheight=240.00mm

\begin{document}
% define the title
\author{Tom Dolan}
\title{First try at \LaTeX}
\date{28/11/2013}
\maketitle

\section{First Problem}
text text text
\[\nabla\cdot\left(\frac{1}{\rho}\nabla P\right) + \frac{\omega^2}{B}P = 0\]
as $\rho$ is a constant it can be taken out of the gradient and multiplied through the equation. 
\[\nabla^2 P + k^2 p = 0\]
\[i.e. \qquad\frac{\partial^2 P}{\partial x^2} + \frac{\partial^2 P}{\partial y^2} + k^2 p = 0\]
If we assume a solution of the form $P(x,y) = X(x)Y(y)$, substituting in to the equation gives:
\[X''Y + XY'' + k^2XY = 0\]
\[\frac{X''}{X} + \frac{Y''}{Y} + k^2 = 0\]
\[\frac{X''}{X} + k^2 = -\frac{Y''}{Y} = \alpha^2\]
\[\text{so } \qquad Y'' + \alpha^2 Y = 0\]
so Y takes the form $Y = Ae^{icy} \implies Y' = Aice^{icy} \implies Y'' = -Ac^2 e^{icy}$
\[\text{so} \qquad -Ac^2e^{icy} + \alpha^2Ae^{icy} = 0\]
\[c^2 = \alpha^2\]
\[c = \pm \alpha\]

\section{giving it another go \ldots}
text text text

\begin{eqnarray*}
\nabla\cdot\left(\frac{1}{\rho}\nabla P\right) + \frac{\omega^2}{B}P & = & 0 \\
\end{eqnarray*}
as $\rho$ is a constant it can be taken out of the gradient and multiplied through the equation. 
\begin{eqnarray*}
\nabla^2 P + k^2 p & = & 0\\
\frac{\partial^2 P}{\partial x^2} + \frac{\partial^2 P}{\partial y^2} + k^2 p & = & 0 \qquad\\
\end{eqnarray*}
If we assume a solution of the form $P(x,y) = X(x)Y(y)$, substituting in to the equation gives:
\begin{eqnarray*}
X''Y + XY'' + k^2XY & = & 0\\
\frac{X''}{X} + \frac{Y''}{Y} + k^2 & = & 0\\
\frac{X''}{X} + k^2 & = &  -\frac{Y''}{Y} = \alpha^2\\
\text{so } \qquad Y'' + \alpha^2 Y & = & 0\\
\end{eqnarray*}
so Y takes the form $Y = Ae^{icy} \implies Y' = Aice^{icy} \implies Y'' = -Ac^2 e^{icy}$
\begin{eqnarray*}
\text{so} \qquad -Ac^2e^{icy} + \alpha^2Ae^{icy} & = & 0\\
c^2 & = & \alpha^2\\
c & = & \pm \alpha\\
\therefore \qquad Y & = & A_1e^{i\alpha y} + A_2e^{-i\alpha y}
\end{eqnarray*}

With Neumann boundary conditions: $P_x(x,0) = P_x(x, L) = 0 \quad  i.e. \quad  X'(x)Y(0) = X'(x)Y(L) = 0$ 

As X'(x) is not necessarily zero for all x this implies $Y(0) = Y(L) = 0$

\begin{eqnarray*}
\text{so} \qquad Y(0)  =  A_1e^{i\alpha 0} + A_2e^{-i\alpha 0} & = & Y(L)  =  A_1e^{i\alpha L} + A_2e^{-i\alpha L} = 0 \\
A_1 + A_2 & = & 0\\
\implies \qquad A_1 & = & -A_2 \quad (set A_1 = A)\\
\text{then} \qquad  Ae^{i\alpha L} - Ae^{-i\alpha L} & = & 0 \\
 Ae^{i\alpha L} & = & Ae^{-i\alpha L}\\
\alpha & = & -\alpha\\
\alpha & = & 0
\end{eqnarray*}

Therefore $Y = A_1 + A_2 = C \quad \forall y \in [0, L]$ 

Which implies $P(x,y) = P(x)$
\pagebreak

\section{restarting\ldots}
\setlength{\unitlength}{4mm}
\begin{picture}(35,5)
\put(2.5,0){\line(1,0){30}}
\put(2.5,5){\line(1,0){30}}
\put(2.5,0.1){\line(1,0){30}}
\put(2.5,4.9){\line(1,0){30}}
\put(5,1){\vector(1,0){3}}
\put(5,1){\vector(0,1){3}}
\put(5.2,4){$y$}
\put(8.2,1){$x$}
\put(33,0){$y = L$}
\put(33,5){$y = 0$}
\end{picture}
\begin{align*}
\nabla\cdot\left(\frac{1}{\rho}\nabla P\right) + \frac{\omega^2}{B}P & = 0 \\
\text{So} \qquad \nabla^2 P + k^2 P & = 0 \\
\frac{\partial^2 P}{\partial x^2} + \frac{\partial^2 P}{\partial y^2} + k^2 P & = 0\\\\
\text {So if $P = X(x)Y(y)$, then }\nabla^2 P & = X''Y + XY''\\
\text{So} \qquad \frac{X''}{X} + \frac{Y''}{Y} + k^2 & = 0\\
i.e. \qquad \frac{X''}{X} + k^2 = -\frac{Y''}{Y} & = \alpha^2\\\\
\text{So we have two 2nd order linear ODEs:}\\
\frac{X''}{X} + k^2 &= \alpha^2\\
X'' + (k^2 - \alpha^2) X & = 0\tag{1}\\ \\
\text{and}\qquad\qquad -\frac{Y''}{Y} & = \alpha^2\\
Y'' + \alpha^2Y & = 0\tag{2}\\\\
\text{Solving (2) gives:}\qquad\\
Y = a_1cos(\alpha y) + a_2sin(\alpha y)\\\\
\text{And given Neumann boundary conditions:}\\
\frac{\partial P}{\partial \overrightarrow{n}} = 0 \quad \forall \bold{x}\in \partial P
\implies & (\frac{\partial P}{\partial x}, \frac{\partial P}{\partial y})\cdot(0,1) = 0\\
\implies & \frac{\partial P}{\partial y} = 0 \quad \text{for } y = 0,L\\
\implies & Y'(0) = Y'(L) = 0\\
\\
Y'(y) = -\alpha a_1\sin(\alpha y) + \alpha a_2\cos(\alpha y)\\
\text{So} \qquad  Y'(0) = \alpha a_2 = 0  \implies & a_2 = 0\\
\text{and} \qquad  Y'(L) =  -\alpha a_1\sin(\alpha y) = 0 \implies & \alpha = n\pi/L \qquad \text{for } n \in \mathbb{N}\\\\
\text{So as there are still unknown constants in X, set } Y_n & = \cos(\frac{n\pi y}{L})
\end{align*}

Now with $\alpha_n  = n\pi/L$, (1) becomes:
\begin{align*}
X'' + (k^2 - \left(\frac{n\pi}{L}\right)^2)X & = 0 \qquad \forall n \in \mathbb{N}\\
\text{so} \qquad X_n & = b_{n+}e^{i\beta_n x} + b_{n-}e^{-i\beta_n x} \qquad \left(\text{where $\beta_n = \sqrt{k^2 - \left(\frac{n\pi}{L}\right)^2}$}\right)\\
\text{so}\qquad  P_n = X_nY_n & = \left(b_{n+}e^{i\beta_n x} + b_{n-}e^{-i\beta_n x}\right)\cos(\alpha_n y)\\
\text{set } k = 1 \implies P_n & = \left(b_{n+}e^{i\sqrt{1 - \left(\frac{n\pi}{L}\right)^2}x} + b_{n-}e^{-i\sqrt{1 - \left(\frac{n\pi}{L}\right)^2}x}\right)\cos(\frac{n\pi y}{L})\\
\end{align*}

The $Y_n$ component of $P_n$ will always be $\cos(n\pi y/L)$ and then if $L>n$ then $\beta_n$ is real so the $X_n$ component takes the form $c_1\cos(\beta_n x) + c_2\sin(\beta_nx)$. \\\\
 If $L< n$ then $\beta_n$ is imaginary so the $X_n$ component takes the form $b_{n+}e^{\gamma x}+ b_{n-}e^{-\gamma x}$. \\\\
\begin{align*}
Y_n & = \cos(\alpha_ny) = \frac{1}{2}\left(e^{i\alpha_ny}+e^{-i\alpha_ny}\right)\\
\text{So } P_n & = X_nY_ne^{-i\omega t}\\
& = \frac{1}{2}\left(b_{n+}e^{i\beta_nx}+b_{n-}e^{-i\beta_nx}\right)\left(e^{i\alpha_ny}+e^{-i\alpha_ny}\right)e^{-i\omega t} \\
& = \frac{1}{2}\Big(b_{n+}e^{i\beta_nx + i\alpha_ny - i\omega t} + b_{n+}e^{i\beta_nx - i\alpha_ny - i\omega t} \\
& \quad+ b_{n-}e^{-i\beta_nx + i\alpha_ny - i\omega t} + b_{n-}e^{-i\beta_nx - i\alpha_ny - i\omega t}\Big) \\
\end{align*}
Now, $\alpha_n = \frac{n\pi}{L}$ and $\beta_n = \sqrt{1-\left(\frac{n\pi}{L}\right)^2} = \sqrt{1-\alpha_n^2}$.  What we want is for terms of $P_n$ to be of the form $ae^{i\cos(\psi)x + i\sin(\psi)y - i\omega t}$ as this is a simple plane wave traveling at an angle of $\psi$ to the x axis
\begin{align*}
\text{so if } \alpha_n & = \sin(\psi_n)\\
\alpha_n^2 & = \sin^2(\psi_n)\\
\alpha_n^2 + \cos^2(\psi_n) & = 1\\
\cos(\psi_n) & = \sqrt{1-\alpha_n^2}\\
& = \beta_n\\\\
\implies P_n & =  \frac{1}{2}\Big(b_{n+}e^{i\cos(\psi_n)x + i\sin(\psi_n)y - i\omega t} + b_{n+}e^{i\cos(\psi_n)x - i\sin(\psi_n)y - i\omega t} \\
& \quad+ b_{n-}e^{-i\cos(\psi_n)x + i\sin(\psi_n)y - i\omega t} + b_{n-}e^{-i\cos(\psi_n)x - i\sin(\psi_n)y - i\omega t}\Big) \\
& = \frac{1}{2}\Big(b_{n+}e^{i\cos(\psi_n)x + i\sin(\psi_n)y - i\omega t} + b_{n+}e^{i\cos(-\psi_n)x + i\sin(-\psi_n)y - i\omega t} \\
& \quad+ b_{n-}e^{-(i\cos(-\psi_n)x + i\sin(-\psi_n)y + i\omega t)} + b_{n-}e^{-(i\cos(\psi_n)x + i\sin(\psi_n)y + i\omega t)}\Big) \\
\end{align*}
So if $b_{n+} = 1$ and $b_{n-} = 0$ then:\\
$P = \frac{1}{2}\Big(e^{i\cos(\psi_n)x + i\sin(\psi_n)y - i\omega t} + e^{i\cos(-\psi_n)x + i\sin(-\psi_n)y - i\omega t}\Big)$ which is the sum of two plane waves traveling at angles of $\psi_n$ and $-\psi_n$ to the x axis. This corresponds to a traveling wave with nodal lines, where $P = 0$ and $t = 0$ at:
\begin{align*}
y &= \frac{\pi(2k-1)}{2\alpha} \qquad \text{for } k = 1, \ldots, n\\
\text{and } x & = \frac{\pi(2l-1)}{2\beta} \qquad \text{for } l \in \mathbb{Z}\\
\end{align*}
As $|e^{i\cos(\psi_n)x + i\sin(\psi_n)y - i\omega t}|$ and $|e^{i\cos(-\psi_n)x + i\sin(-\psi_n)y - i\omega t}|$ are bounded by 1, half their sum, i.e. $|P|$ is bounded by 1. 
\end{document}}

