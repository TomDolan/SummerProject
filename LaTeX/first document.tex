\documentclass[a4paper,11pt]{article}
\usepackage{amsmath}
\usepackage{amssymb}
\oddsidemargin=-5.00mm
\textwidth=170.00mm
\topmargin=-10.00mm
\textheight=240.00mm

\begin{document}
% define the title
\author{Tom Dolan}
\title{First try at \LaTeX}
\date{28/11/2013}
\maketitle

\section{First Problem}
text text text
\[\nabla\cdot\left(\frac{1}{\rho}\nabla P\right) + \frac{\omega^2}{B}P = 0\]
as $\rho$ is a constant it can be taken out of the gradient and multiplied through the equation. 
\[\nabla^2 P + k^2 p = 0\]
\[i.e. \qquad\frac{\partial^2 P}{\partial x^2} + \frac{\partial^2 P}{\partial y^2} + k^2 p = 0\]
If we assume a solution of the form $P(x,y) = X(x)Y(y)$, substituting in to the equation gives:
\[X''Y + XY'' + k^2XY = 0\]
\[\frac{X''}{X} + \frac{Y''}{Y} + k^2 = 0\]
\[\frac{X''}{X} + k^2 = -\frac{Y''}{Y} = \alpha^2\]
\[\text{so } \qquad Y'' + \alpha^2 Y = 0\]
so Y takes the form $Y = Ae^{icy} \implies Y' = Aice^{icy} \implies Y'' = -Ac^2 e^{icy}$
\[\text{so} \qquad -Ac^2e^{icy} + \alpha^2Ae^{icy} = 0\]
\[c^2 = \alpha^2\]
\[c = \pm \alpha\]

\section{giving it another go \ldots}
text text text

\begin{eqnarray*}
\nabla\cdot\left(\frac{1}{\rho}\nabla P\right) + \frac{\omega^2}{B}P & = & 0 \\
\end{eqnarray*}
as $\rho$ is a constant it can be taken out of the gradient and multiplied through the equation. 
\begin{eqnarray*}
\nabla^2 P + k^2 p & = & 0\\
\frac{\partial^2 P}{\partial x^2} + \frac{\partial^2 P}{\partial y^2} + k^2 p & = & 0 \qquad\\
\end{eqnarray*}
If we assume a solution of the form $P(x,y) = X(x)Y(y)$, substituting in to the equation gives:
\begin{eqnarray*}
X''Y + XY'' + k^2XY & = & 0\\
\frac{X''}{X} + \frac{Y''}{Y} + k^2 & = & 0\\
\frac{X''}{X} + k^2 & = &  -\frac{Y''}{Y} = \alpha^2\\
\text{so } \qquad Y'' + \alpha^2 Y & = & 0\\
\end{eqnarray*}
so Y takes the form $Y = Ae^{icy} \implies Y' = Aice^{icy} \implies Y'' = -Ac^2 e^{icy}$
\begin{eqnarray*}
\text{so} \qquad -Ac^2e^{icy} + \alpha^2Ae^{icy} & = & 0\\
c^2 & = & \alpha^2\\
c & = & \pm \alpha\\
\therefore \qquad Y & = & A_1e^{i\alpha y} + A_2e^{-i\alpha y}
\end{eqnarray*}

With Neumann boundary conditions: $P_x(x,0) = P_x(x, L) = 0 \quad  i.e. \quad  X'(x)Y(0) = X'(x)Y(L) = 0$ 

As X'(x) is not necessarily zero for all x this implies $Y(0) = Y(L) = 0$

\begin{eqnarray*}
\text{so} \qquad Y(0)  =  A_1e^{i\alpha 0} + A_2e^{-i\alpha 0} & = & Y(L)  =  A_1e^{i\alpha L} + A_2e^{-i\alpha L} = 0 \\
A_1 + A_2 & = & 0\\
\implies \qquad A_1 & = & -A_2 \quad (set A_1 = A)\\
\text{then} \qquad  Ae^{i\alpha L} - Ae^{-i\alpha L} & = & 0 \\
 Ae^{i\alpha L} & = & Ae^{-i\alpha L}\\
\alpha & = & -\alpha\\
\alpha & = & 0
\end{eqnarray*}

Therefore $Y = A_1 + A_2 = C \quad \forall y \in [0, L]$ 

Which implies $P(x,y) = P(x)$

\end{document}}
